% Preamble
% ---
\documentclass{article}

% Packages
% ---
\usepackage{amsmath} % Advanced math typesetting
\usepackage[utf8]{inputenc} % Unicode support (Umlauts etc.)
\usepackage{hyperref} % Add a link to your document
\usepackage{graphicx} % Add pictures to your document
\usepackage{subfig}
\usepackage{listings} % Source code formatting and highlighting
\usepackage{float}
\usepackage[margin=2cm]{geometry}


\title{Guía de uso de Lattice Radiant Software}
\date{2019-Octubre}
\author{Electrónica 3}

\begin{document}

\maketitle
\pagenumbering{gobble}
\newpage
\pagenumbering{arabic}

\tableofcontents
\pagebreak

%Empiezan las secciones del articulo

\section{Descarga e instalación}
El software utilizado para programar la FPGA provista por la cátedra es 'Lattice Radiant Software'.El mismo puede descargarse tanto para Linux como para Windows del siguiente link: \url{https://tinyurl.com/y46mth4j}
\subsection{ Elección del sistema operativo}

\begin{figure}[H]
\centering
\includegraphics[width=0.4\linewidth]{Imagenes/1.JPG}
\caption{Guía de descarga: Paso 1}
\label{fig:step1}
\end{figure}


\subsection{ Ejemplo de selección: caso Windows}

\begin{figure}[H]
\centering
\includegraphics[width=1\linewidth]{Imagenes/2.JPG}
\caption{Guía de descarga: Paso 2}
\label{fig:step2}
\end{figure}

\subsection{Sign in request}
Después de haber seleccionado Lattice Radiant Software, nos lleva una página donde para poder entrar tenemos que estar registrados.

\begin{figure}[H]
\centering
\includegraphics[width=0.4\linewidth]{Imagenes/3.JPG}
\caption{Guía de descarga: Paso 3}
\label{fig:step3}
\end{figure}

Para poder seguir, procedemos a registrarnos

\subsection{Registration}
A continuación dejamos una captura de cómo hay que rellenar ciertos campos específicos. Sugerimos utilizar el mail del itba a la hora de registrarse.

\begin{figure}[H]
\centering
\includegraphics[width=0.4\linewidth]{Imagenes/4.JPG}
\caption{Guía de descarga: Paso 4}
\label{fig:step4}
\end{figure}

\subsection{Descarga}
Ahora que tenemos una cuenta, volvemos a la página de lattice para proceder con la descarga. En esa página, aceptamos términos y condiciones, y a continuación clickeamos en Download.

\begin{figure}[H]
\centering
\includegraphics[width=1\linewidth]{Imagenes/5.JPG}
\caption{Guía de descarga: Paso 5}
\label{fig:step5}
\end{figure}

\subsection{Instalación}
\subsubsection{Ejecutar el archivo .exe}

Descomprimimos el archivo descargado y ejecutamos el archivo .exe que se encuentra dentro de la carpeta
\begin{figure}[H]
\centering
\includegraphics[width=0.1\linewidth]{Imagenes/zip.PNG}
\caption{Archivo a descomprimir}
\label{fig:install}
\end{figure}

\newpage

\subsubsection{Capturas de instalación}

\begin{figure}[H]
\centering
\includegraphics[width=0.8\linewidth]{Imagenes/inst1.JPG}
\caption{Instalación: captura 1 }
\label{fig:install}
\end{figure}

\begin{figure}[H]
\centering
\includegraphics[width=0.8\linewidth]{Imagenes/inst2.JPG}
\caption{Instalación: captura 2 }
\label{fig:install}
\end{figure}

\begin{figure}[H]
\centering
\includegraphics[width=0.8\linewidth]{Imagenes/inst3.JPG}
\caption{Instalación: captura 3 }
\label{fig:install}
\end{figure}

\begin{figure}[H]
\centering
\includegraphics[width=0.8\linewidth]{Imagenes/inst4.JPG}
\caption{Instalación: captura 4 }
\label{fig:install}
\end{figure}

\begin{figure}[H]
\centering
\includegraphics[width=0.8\linewidth]{Imagenes/inst5.JPG}
\caption{Instalación: captura 5 }
\label{fig:install}
\end{figure}

\begin{figure}[H]
\centering
\includegraphics[width=0.8\linewidth]{Imagenes/inst6.JPG}
\caption{Instalación: captura 6 }
\label{fig:install}
\end{figure}


\begin{figure}[H]
\centering
\includegraphics[width=0.8\linewidth]{Imagenes/inst7.JPG}
\caption{Instalación: captura 7 }
\label{fig:install}
\end{figure}


\begin{figure}[H]
\centering
\includegraphics[width=0.8\linewidth]{Imagenes/inst8.JPG}
\caption{Instalación: captura 8 }
\label{fig:install}
\end{figure}



\subsection{Licencia}
Para tener una licencia, hay que generar una solicitud en
\href{https://www.latticesemi.com/Support/Licensing}{https://www.latticesemi.com/Support/Licensing}.
En esa página, vamos la parte de Lattice Radiant Software y clickeamos en Request a Free License.

\begin{figure}[H]
\centering
\includegraphics[width=1\linewidth]{Imagenes/6.JPG}
\caption{Guía de descarga: Paso 6}
\label{fig:step6}
\end{figure}

El proceso puede demorar 1 día así que hay que procurar no hacerlo a último momento.

\section{Creación de un proyecto}
	Para crear un nuevo proyecto se debe abrir el programa recientemente instalado y elegir 'New Project'.
	\begin{figure}[H]
	\centering
	\includegraphics[height=6cm,width=10cm]{Imagenes/NewProj.png}
	\caption{Opción para crear un nuevo proyecto}
	\end{figure}
	
	Luego de hacer click en New Project y en el botón de next, se debe elegir el nombre del proyecto y en que carpeta se desea guardar.Dejar el nombre bajo el campo de "Implementation" en su valor default y hacer click en next nuevamente.
	En la siguiente ventana se puede elegir agregar archivos al nuevo proyecto.En este paso se puede elegir agregar cualquier archivo de Verilog ya existente que sea necesario para el proyecto.
	\begin{figure}[H]
	\centering
	\includegraphics[height=6cm,width=8cm]{Imagenes/AddSources.png}
	\caption{Ventana para agregar archivos ya existentes}
	\end{figure}
	
	Tildar las opciones como se indica en la figura anterior y hacer click en Next.
	En la siguiente ventana se indica el dispositivo a utilizar, completar las opciones igual que como se muestra en la siguiente figura:
	\begin{figure}[H]
	\centering
	\includegraphics[height=10cm,width=10cm]{Imagenes/Dispositivo.png}
	\caption{Prestar especial antencion a que el campo de 'Package' y 'Part Number' coincidan con el de la imagen }
	\end{figure}
	Clickear Next nuevamente, elegir la opción 'Lattice LSE' en la siguiente ventana, elegir next una vez mas y luego Finish.
	

\subsection{Módulos de Verilog}
Se pueden crear/agregar archivos de Verilog al proyecto seleccionando 'File' arriba a la izquierda y luego 'New' para crear un archivo o 'Add' para agregar uno ya existente.

Los archivos de Verilog del proyecto se pueden ver dentro de la carpeta llamada 'Input files'
\begin{figure}[H]
	\centering
	\includegraphics[height=8cm,width=6cm]{Imagenes/VerilogArch.png}
	\caption{Ubicación de los proyectos de Verilog}
	\end{figure}

\subsection{Time constraints}
Para establecer los clocks necesarios para el proyecto se puede hacer uso del 'Timing Constraint Editor'.
	\begin{figure}[H]
	\centering
	\includegraphics[height=1cm,width=\linewidth]{Imagenes/UbicacionTiming.png}
	\caption{Ubicación del 'Timing constraint editor'}
	\end{figure}
Una vez en la ventana del editor se debe seleccionar el puerto de entrada que se desea como clock y elegir la frecuencia deseada para el mismo.
	\begin{figure}[H]
	\centering
	\includegraphics[width=0.7\linewidth]{Imagenes/Clk_ejemplo.png}
	\caption{Ejemplo para generar un clock de 12MHz en el puerto 'clk'}
	\end{figure}
Para el puerto a utilizar como clock asignarle al mismo el pin 35 de la FPGA.

\section{Simulación}

\section{Asignación de pins}
Para asignar que pin de la FPGA corresponde a que entrada y salida del modulo de Verilog, se debe ir al 'Device Constraint Editor'.
	\begin{figure}[H]
	\centering
	\includegraphics[width=\textwidth]{Imagenes/pins.png}
	\caption{Ubicación del Device Constraint Editor en el Radiant}
	\end{figure}
	
 Una vez abierto el Device Constraint Editor se vera algo similar a lo observado en la siguiente figura:
 	\begin{figure}[H]
 	\centering
	\includegraphics[height=9cm,width=16cm]{Imagenes/DeviceConstraintEditor.png}
	\caption{Vista de las señales de Verilog y sus pins correspondientes}
	\end{figure}
 En esta ventana se puede cambiar que pin corresponde a una señal del modulo de Verilog mas alto en la jerarquía.Para realizar cambios solo hace falta hacer click en el campo de 'pin' correspondiente a una señal dada y cambiar el valor numérico.Antes del nombre de cada señal hay una flecha con una dirección y color determinado que indica si la señal es de input o de output.
 
 Hay que tener especial cuidado de asignar pins validos para las señales (utilizar los pins I/O).A continuación se presentan algunas tablas con la función de algunos pins de la FPGA que utiliza la cátedra(ICE40-UP5K).
 \begin{figure}[H]
 \centering
 \subfloat{\includegraphics[height=12cm,width=8cm]{Imagenes/SidePins.png}\label{fig:f1}}
 \subfloat{\includegraphics[height=10cm, width=7cm]{Imagenes/FrontPins.png}\label{fig:f2}}
	\end{figure}
	

\section{Compilación,síntesis y programador}
Para cargar el programa en Verilog a la FPGA se deben seguir los pasos de síntesis, mappeo, routeo y exportación que aparecen arriba de la ventana de archivos.
	\begin{figure}[H]
 	\centering
	\includegraphics[height=8cm, width=\textwidth]{Imagenes/SintesisRouteo.png}
	\end{figure}
Se puede realizar cada uno de dichos pasos uno por vez o se puede elegir realizar todos clickeando el símbolo verde de play a la izquierda. Ya en el proceso de síntesis el compilador detectara posibles errores de sintaxis en el código de Verilog así como también los warnings.

Si se desea ver en detalle los resultados de cada paso se puede ir a 'view' en la parte de arriba y luego seleccionar reports.
	\begin{figure}[H]
 	\centering
	\includegraphics[height=8cm, width=10cm]{Imagenes/reports.png}
	\caption{Ventana con los reportes con cada uno de los pasos efectuados}
	\end{figure}
	
Una vez que se consiguieron exitosamente los archivos de salida solo queda cargar los mismos a la FPGA.Para este paso se requiere primero conectar la FPGA a la computadora mediante un cable USB a Micro USB.Luego de conectar la FPGA se debe abrir el programmer el cual esta ubicado en la parte superior derecha del lattice como se indica en la siguiente figura.
	\begin{figure}[H]
 	\centering
	\includegraphics[height=2cm, width=\textwidth]{Imagenes/ProgrammerUbi.png}
	\caption{Ubicación del programmer}
	\end{figure}

Al abrir el programmer con la FPGA ya conectada debería verse algo similar a la siguiente figura:
	\begin{figure}[H]
 	\centering
	\includegraphics[height=6cm, width=8cm]{Imagenes/Programmer.png}
	\caption{Ventana luego de abrir el programmer}
	\end{figure}
Como se puede ver en la figura anterior debería aparecer listada la FPGA que se conecto a la PC. Se puede seleccionar la opción 'Detect Cable' a la derecha para verificar que la computadora reconoció exitosamente la FPGA. Si se tiene problemas con este paso se puede intentar esperar un poco y luego volver a intentar con 'Detect Cable', si esto no funciona se puede probar conectando la FPGA a otro puerto USB de la PC.Por si acaso siempre es mejor conectar y desconectar la FPGA con el programador cerrado y luego volver a abrirlo.

Ahora solo quedo configurar la FPGA.Para eso se debe seleccionar el dispositivo, luego edit y de ahí Device Properties.
	\begin{figure}[H]
 	\centering
	\includegraphics[height=6cm, width=8cm]{Imagenes/EditProg.png}
	\end{figure}
Configurar la ventana igual que como se muestra en la siguiente figura:
	\begin{figure}[H]
 	\centering
	\includegraphics[height=10cm, width=10cm]{Imagenes/Config.png}
	\end{figure}
Para el programming file elegir el archivo con la extension '.bin' resultante de la exportación que se realizo antes de abrir el programmer.Finalmente solo queda subir el .bin a la FPGA seleccionando la opción 'Program Device' que aparece en la barra de arriba.
	\begin{figure}[H]
 	\centering
	\includegraphics[height=2cm, width=\textwidth]{Imagenes/runprog.png}
	\end{figure}

\end{document}